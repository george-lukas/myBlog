% Created 2016-04-28 Thu 18:23
\documentclass[11pt]{article}
\usepackage[latin1]{inputenc}
\usepackage[T1]{fontenc}
\usepackage{fixltx2e}
\usepackage{graphicx}
\usepackage{grffile}
\usepackage{longtable}
\usepackage{wrapfig}
\usepackage{rotating}
\usepackage[normalem]{ulem}
\usepackage{amsmath}
\usepackage{textcomp}
\usepackage{amssymb}
\usepackage{capt-of}
\usepackage{hyperref}
\author{George Lukas}
\date{\today}
\title{}
\hypersetup{
 pdfauthor={George Lukas},
 pdftitle={},
 pdfkeywords={},
 pdfsubject={},
 pdfcreator={Emacs 24.5.1 (Org mode 8.3.4)}, 
 pdflang={English}}
\begin{document}

\tableofcontents

\section{Introdu��o}
\label{sec:orgheadline6}
\begin{verse}
Neste tutorial faremos um blog usando o Yesod Frameword.\\
Yesod � um framework web Haskell para o desenvolvimento produtivo de, RESTful, aplica��o web de tipo\\
seguro e de alta performance.\\
\end{verse}

\subsection{Come�ando..}
\label{sec:orgheadline2}
\subsubsection{Pacotes a serem instalados}
\label{sec:orgheadline1}
\begin{itemize}
\item psql (postgres)
\item stack
\item yesod
\item yesod-bin
\end{itemize}
\subsection{Criando um novo projeto}
\label{sec:orgheadline3}
\begin{verse}
Dado que o psql e o stack foi devidamente instalado, criamos um novo projeto usando o Stack, cuja a\\
sintaxe � \emph{stack new <project name> <template>}.\\
Para esse caso:\\
\end{verse}
\begin{itemize}
\item \emph{stack new myBlog yesod-sqlite \&\& cd myBlog}
\end{itemize}

\begin{verse}
Para construir as bibliotecas\\
\end{verse}
\begin{itemize}
\item \emph{stack exec -- yesod devel}
\end{itemize}

\begin{quote}
Agora voc� pode testar sua aplica�ando em \url{http://localhost:3000/} \\
Ap�s isso, criamos uma sandbox para nosso projeto usando o Cabal: 
\end{quote}
\begin{itemize}
\item \emph{cabal sandbox init}
\end{itemize}

\begin{quote}
Agora instalamos apenas as dependencias necess�rias para nossa aplica��o:
\end{quote}

\begin{itemize}
\item \emph{cabal install --only-dependencies}
\end{itemize}

\begin{quote}
Em seguida podemos criar nossa aplica��o e test�-la:
\end{quote}

\begin{itemize}
\item \emph{cabal build}
\item \emph{yesod-devel}
\end{itemize}

\subsection{Criando Rotas}
\label{sec:orgheadline4}

\begin{quote}
Antes de irmos a frente, criamos um novo git repo.\\
\end{quote}

\begin{itemize}
\item \emph{git init}
\item \emph{git add .}
\item \emph{git commit -m "myBlog" .}
\end{itemize}

\begin{quote}
Agora estamos prontos para seguir em frente.\\
A primeira coisa que temos que fazer para uma aplica��o de blog, � uma p�gina que nos permita criar blog posts.\\
Uma p�gina que nos permita criar amostras de dados e ver como o yesod cria rotas. N�o faremos tudo na m�o,\\
ent�o no terminal:
\end{quote}

\begin{itemize}
\item \emph{yesod add-handler}
\end{itemize}

\begin{quote}
Isto abrir� um menu, para esse caso respondemos:
\end{quote}

\begin{itemize}
\item PostNew
\item /post/new
\item GET
\end{itemize}

\begin{quote}
Podemos ver que houve algumas mudan�as com o comando:\\
\begin{itemize}
\item \emph{git status}
\end{itemize}
\end{quote}

\begin{quote}
Agora estamos prontos para continuar com este Handler.\\
Handler/PostNew.hs e editamos o que est� l� para:
\end{quote}

\begin{verbatim}
module Handler.PostNew where

import Import

getPostNewR :: Handler Html
getPostNewR = do
  defaultLayout $ do
    $(widgetFile "posts/new")
\end{verbatim}

\begin{quote}
Em seguida criaremos uma nova p�gina, um template. Criamos uma nova pasta dentro de templates.\\
templates/posts. A seguir criamos um documento em equivalente ao html, o hamlet.\\
\textbf{new.hamlet}\\
Para testarmos, editamos para:
\end{quote}

\begin{verbatim}
<h1> Create a post!
\end{verbatim}

Interprete o Handler e pode-se testar no browser.
\url{http://localhost:3000/posts/new}

"Insert copyright statement here" n�o � necess�riamente requerido. Isso vem da fun��o \emph{defaultLayout}
que chama o dafault-layout-wrapper.hamlet e l� h� um campo que est� especificado o seguinte:

\begin{verbatim}
<body>
   <div class="container">
     <header>
     <div id="main" role="main">
       ^{pageBody pc}
     <footer>
       #{appCopyright $ appSettings master}
\end{verbatim}

Ent�o � s� apagar a seguinte linha:
\begin{quote}
\#\{appCopyright \$ appSettings master\}
\end{quote}

\subsection{Criando Forms}
\label{sec:orgheadline5}
Fazemos algumas mudan�as no Handler:

\begin{verbatim}
module Handler.PostNew where

import Import
import Yesod.Form.Bootstrap3

data BlogPost = BlogPost
	{ title :: Text
	}

blogPostForm :: AForm Handler BlogPost
blogPostForm = BlogPost
	    <$> areq textField (bfs ("Title" :: Text)) Nothing

getPostNewR :: Handler Html
getPostNewR = do
    (widget, enctype) <- generateFormPost $ renderBootstrap3 BootstrapBasicForm blogPostForm
    defaultLayout $ do
	$(widgetFile "posts/new")
\end{verbatim}

E tamb�m algumas mudan�as no template:

\begin{verbatim}
<h1> Create a post!

<form method=post action=@{PostNewR} enctype=#{enctype}
  ^{widget}
  <button class="btn btn-default">Create Post!
\end{verbatim}

Agora precisamos de algo para gerar novos blogposts. Adicionaremos dois pacotes em nosso myBlog.cabal.
Na guia \textbf{build-depends}, ao fim dela adicione:
\begin{quote}
, markdown \\
, yesod-text-markdown
\end{quote}
\end{document}
